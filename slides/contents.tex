%%%%%%%%%%%%%%%%%%%%%%%%%%%%%%%%%%%%%%%%%%%%%%%%%%
% CoolBeans
\begin{frame}{Problem: How to stub indirect dependencies (that stem from possibly nested "require"s)?}
\begin{itemize}
\item x
\end{itemize}
\end{frame}

%%%%%%%%%%%%%%%%%%%%%%%%%%%%%%%%%%%%%%%%%%%%%%%%%%
% nconf
\begin{frame}{Problem: We configure our application via a configuration framework. How can we provide a standardized test configuration with minimal setup?}
\begin{itemize}
\item x
\end{itemize}
\end{frame}


%%%%%%%%%%%%%%%%%%%%%%%%%%%%%%%%%%%%%%%%%%%%%%%%%%
% 
\begin{frame}{Problem: How to adapt to various testing needs?}
\begin{itemize}
\item x
\end{itemize}
\end{frame}


%%%%%%%%%%%%%%%%%%%%%%%%%%%%%%%%%%%%%%%%%%%%%%%%%%
% UI tests / Application tests
\begin{frame}{Problem: How can I perform UI tests with only setting up a minimal part of  the application?}
\begin{itemize}
\item x
\end{itemize}
\end{frame}


%%%%%%%%%%%%%%%%%%%%%%%%%%%%%%%%%%%%%%%%%%%%%%%%%%
% DB tests
% vorher: nconf
\begin{frame}{Problem: How to test a database in integration?}
\begin{itemize}
\item We use native queries and other database features in our code.
\item In our test configuration, we replace the module that connects to the database by a fake.
\end{itemize}

\begin{itemize}
\item Question: How to test the database integration?
\item Question: How to test native queries?
\end{itemize}
\end{frame}

\begin{frame}{Our Solution}
\begin{itemize}
\item We set up a second test configuration where we connect to a test database (i.e.~a special collection in our mongoDB).
\item Nice: We cannot mess up production data even if we run the tests on the production system.
\end{itemize}
\end{frame}

%This is our configuration for tests with DB:
%https://github.com/softwerkskammer/Agora/blob/master/testutil/configureForTestWithDB.js
%It uses the following beans definitions for the modules that connect to the database:
%https://github.com/softwerkskammer/Agora/blob/master/config/testbeansWithDB.json
%So far, only groupsPersistence, activitiesPersistence and sympaPersistence are being connected to the database; the others still point to the persistence fake because there are currently no integration tests for them.
%
%The actual database tests are now pretty straightforward:
%https://github.com/softwerkskammer/Agora/blob/master/testWithDB/activitiesDB/activities_index_upcoming_past_test.js
%In the setup, clear the database, persist the desired objects, run the test and check the results.


%%%%%%%%%%%%%%%%%%%%%%%%%%%%%%%%%%%%%%%%%%%%%%%%%%
% Race conditions
% vorher: CoolBeans
% vorher: Application tests
% vorher: integrated db tests
\begin{frame}{Problem: How to test race conditions on a database?}
\begin{itemize}
\item To prevent data loss due to concurrent modification of data, we use version counters.
\item If feasible, we apply automatic retries. 
\item Otherwise, the user is asked to redo his operation.
\end{itemize}
\end{frame}

\begin{frame}{Question}
\begin{itemize}
\item How can we test that 
\begin{itemize}
\item there will be no data loss due to race conditions?
\item the automatic retry succeeds?
\end{itemize}
\end{itemize}
\end{frame}

\begin{frame}{Our Solution (No-data-loss test)}
\begin{itemize}
\item Actual race condition (i.e.~interleaved load-modify-save sequence) cannot be established in automated test
\item Trick:
\begin{itemize}
\item stub the load operation so that it returns the object in question \textbf{before} the concurrent modification
\item initialize database with object in question \textbf{after} the concurrent modification
\item this simulates that the concurrent modification takes place after the load operation but before the save operation that is performed in our test
\end{itemize}
\item Test checks that the concurrent modification is still present in the object in question after the attempted save operation.
\end{itemize}
\end{frame}

\begin{frame}{Our Solution (Automatic-retry test)}
\begin{itemize}
\item To test the automatic retry:
\begin{itemize}
\item stub the load operation so that it returns the object in question \textbf{before} the concurrent modification on the first invocation
\item and the object in question \textbf{after} the concurrent modification on the second invocation (i.e.~when automatic retry is triggered)
\end{itemize}
\item Test checks that the modification from our test as well as the concurrent modification is present in the object in question after the save operation.
\end{itemize}
\end{frame}


%%%%%%%%%%%%%%%%%%%%%%%%%%%%%%%%%%%%%%%%%%%%%%%%%%
% DOM tests
\begin{frame}{Problem: How can I test frontend code that requires a DOM if all I have is jade template files (but no html files)?}
\begin{itemize}
\item x
\end{itemize}
\end{frame}


%%%%%%%%%%%%%%%%%%%%%%%%%%%%%%%%%%%%%%%%%%%%%%%%%%
% unsolved
\begin{frame}{Problem: Coverage killt Code, der woanders gebraucht wird}
\begin{itemize}
\item x
\end{itemize}
\end{frame}


%%%%%%%%%%%%%%%%%%%%%%%%%%%%%%%%%%%%%%%%%%%%%%%%%%
% unsolved
\begin{frame}{Problem: Wie kann man Code Coverage aus mehreren Testläufen zusammenfassen?}
\begin{itemize}
\item x
\end{itemize}
\end{frame}



% TEMPLATE:
%%%%%%%%%%%%%%%%%%%%%%%%%%%%%%%%%%%%%%%%%%%%%%%%%%
%\begin{frame}{x}
%\begin{itemize}
%\item x
%\end{itemize}
%\end{frame}
%
%%%%%%%
%\note{
%\begin{description}
%\item[N] x
%\item[J] x
%\end{description}
%}


%%%%%%%%%%%%%%%%%%%%%%%%%%%%%%%%%%%%%%%%%%%%%%%%%%
{
\usebackgroundtemplate{\includegraphics[width=\paperwidth,height=\paperheight]{background-slide.png}}
\begin{frame}{Thank You!}

        Slides on GitHub:
        \vspace{-0.8em}
        \begin{center}
                \url{https://github.com/NicoleRauch/}
        \end{center}

     \begin{block}{Andreas Leidig}
        \begin{description}[Twitterxx]
                \item[E-Mail]  \href{mailto:andreas.leidig@msg-gillardon.de}{\texttt{andreas.leidig@msg-gillardon.de}}
                \item[Twitter] \href{http://twitter.com/leiderleider}{\texttt{@leiderleider}}
        \end{description}
      \end{block}
      \begin{block}{Nicole Rauch}
        \begin{description}[Twitterxx]
                \item[E-Mail]  \href{mailto:nicole.rauch@msg-gillardon.de}{\texttt{nicole.rauch@msg-gillardon.de}}
                \item[Twitter] \href{http://twitter.com/NicoleRauch}{\texttt{@NicoleRauch}}
        \end{description}
        \end{block}
\end{frame}
}
